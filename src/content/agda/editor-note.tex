% !TEX root = ctfp-print.tex

%% Edit the next two lines as appropriate for each new language.
\newcommand{\optlang}{Agda\xspace}
\newcommand{\contributors}{\urlref{https://formalverification.io}{formalverification.io}\xspace}
\newcommand{\optlangfig}{fig/icons/agda.png}
\newcommand{\optlangcode}{\srcsnippet{content/3.6/code/agda/snippet03.agda}{purple}{agda}}

\lettrine[lhang=0.17]{T}{his is the
\optlang
edition} of \emph{Category Theory for Programmers}.
It's been a tremendous success, making Bartosz Milewski's blog post series available as a nicely
typeset \acronym{PDF}, as well as a hardcover book. There have been numerous contributions made
to improve the book, by fixing typos and errors, as well as translating the code snippets into
other programming languages.

I am thrilled to present this edition of the book, containing the original Haskell code, followed by its
\optlang
counterpart. The \optlang code snippets were generously provided by
\contributors
contributors, slightly modified to suit the format of this book.

To support code snippets in multiple languages, I am using a \LaTeX{} macro to load the code snippets
from external files. This allows easily extending the book with other languages, while leaving the
original text intact. Which is why you should mentally append the words ``and \optlang'' whenever you see
``in Haskell'' in the text.

The code is laid out in the following manner: the original Haskell code, followed by \optlang code.
To distinguish between them, the code snippets are braced from the left with a vertical bar, in the primary
color of the language's logo, \raisebox{-.2mm}{\includegraphics[height=.3cm]{fig/icons/haskell.png}},
and \raisebox{-.2mm}{\includegraphics[height=.3cm]{\optlangfig}} respectively, e.g.:

\srcsnippet{content/3.6/code/haskell/snippet03.hs}{blue}{haskell}
\unskip
\optlangcode
\NoIndentAfterThis
% In addition, some Scala snippets make use of the
% \urlref{https://github.com/non/kind-projector}{Kind Projector} compiler plugin, to support nicer syntax for partially-applied types.
