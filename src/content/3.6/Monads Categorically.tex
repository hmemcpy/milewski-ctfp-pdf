\begin{quote}
This is part 22 of Categories for Programmers. Previously:
\href{https://bartoszmilewski.com/2016/11/30/monads-and-effects/}{Monads
and Effects}. See the
\href{https://bartoszmilewski.com/2014/10/28/category-theory-for-programmers-the-preface/}{Table
of Contents}.
\end{quote}

If you mention monads to a programmer, you'll probably end up talking
about effects. To a mathematician, monads are about algebras. We'll talk
about algebras later --- they play an important role in programming ---
but first I'd like to give you a little intuition about their relation
to monads. For now, it's a bit of a hand-waving argument, but bear with
me.

Algebra is about creating, manipulating, and evaluating expressions.
Expressions are built using operators. Consider this simple expression:

\begin{verbatim}
x2 + 2 x + 1
\end{verbatim}

This expression is formed using variables like \texttt{x}, and constants
like 1 or 2, bound together with operators like plus or times. As
programmers, we often think of expressions as trees.

\includegraphics[width=1.82292in]{images/exptree.png}

Trees are containers so, more generally, an expression is a container
for storing variables. In category theory, we represent containers as
endofunctors. If we assign the type \texttt{a} to the variable
\texttt{x}, our expression will have the type \texttt{m\ a}, where
\texttt{m} is an endofunctor that builds expression trees. (Nontrivial
branching expressions are usually created using recursively defined
endofunctors.)

What's the most common operation that can be performed on an expression?
It's substitution: replacing variables with expressions. For instance,
in our example, we could replace \texttt{x} with \texttt{y\ -\ 1} to
get:

\begin{verbatim}
(y - 1)2 + 2 (y - 1) + 1
\end{verbatim}

Here's what happened: We took an expression of type \texttt{m\ a} and
applied a transformation of type \texttt{a\ -\textgreater{}\ m\ b}
(\texttt{b} represents the type of \texttt{y}). The result is an
expression of type \texttt{m\ b}. Let me spell it out:

\begin{verbatim}
m a -> (a -> m b) -> m b
\end{verbatim}

Yes, that's the signature of monadic bind.

That was a bit of motivation. Now let's get to the math of the monad.
Mathematicians use different notation than programmers. They prefer to
use the letter \texttt{T} for the endofunctor, and Greek letters: μ for
\texttt{join} and η for \texttt{return}. Both \texttt{join} and
\texttt{return} are polymorphic functions, so we can guess that they
correspond to natural transformations.

Therefore, in category theory, a monad is defined as an endofunctor
\texttt{T} equipped with a pair of natural transformations μ and η.

μ is a natural transformation from the square of the functor \texttt{T2}
back to \texttt{T}. The square is simply the functor composed with
itself, \texttt{T\ ∘\ T} (we can only do this kind of squaring for
endofunctors).

\begin{verbatim}
μ :: T2 -> T
\end{verbatim}

The component of this natural transformation at an object \texttt{a} is
the morphism:

\begin{verbatim}
μa :: T (T a) -> T a
\end{verbatim}

which, in \emph{Hask}, translates directly to our definition of
\texttt{join}.

η is a natural transformation between the identity functor \texttt{I}
and \texttt{T}:

\begin{verbatim}
η :: I -> T
\end{verbatim}

Considering that the action of \texttt{I} on the object \texttt{a} is
just \texttt{a}, the component of η is given by the morphism:

\begin{verbatim}
ηa :: a -> T a
\end{verbatim}

which translates directly to our definition of \texttt{return}.

These natural transformations must satisfy some additional laws. One way
of looking at it is that these laws let us define a Kleisli category for
the endofunctor \texttt{T}. Remember that a Kleisli arrow between
\texttt{a} and \texttt{b} is defined as a morphism
\texttt{a\ -\textgreater{}\ T\ b}. The composition of two such arrows
(I'll write it as a circle with the subscript \texttt{T}) can be
implemented using μ:

\begin{verbatim}
g ∘T f = μc ∘ (T g) ∘ f
\end{verbatim}

where

\begin{verbatim}
f :: a -> T b g :: b -> T c
\end{verbatim}

Here \texttt{T}, being a functor, can be applied to the morphism
\texttt{g}. It might be easier to recognize this formula in Haskell
notation:

\begin{verbatim}
f >=> g = join . fmap g . f
\end{verbatim}

or, in components:

\begin{verbatim}
(f >=> g) a = join (fmap g (f a))
\end{verbatim}

In terms of the algebraic interpretation, we are just composing two
successive substitutions.

For Kleisli arrows to form a category we want their composition to be
associative, and η\textsubscript{a} to be the identity Kleisli arrow at
\texttt{a}. This requirement can be translated to monadic laws for μ and
η. But there is another way of deriving these laws that makes them look
more like monoid laws. In fact \texttt{μ} is often called
multiplication, and \texttt{η} unit.

Roughly speaking, the associativity law states that the two ways of
reducing the cube of \texttt{T}, \texttt{T3}, down to \texttt{T} must
give the same result. Two unit laws (left and right) state that when
\texttt{η} is applied to \texttt{T} and then reduced by \texttt{μ}, we
get back \texttt{T}.

Things are a little tricky because we are composing natural
transformations and functors. So a little refresher on horizontal
composition is in order. For instance, \texttt{T3} can be seen as a
composition of \texttt{T} after \texttt{T2}. We can apply to it the
horizontal composition of two natural transformations:

\begin{verbatim}
IT ∘ μ
\end{verbatim}

\includegraphics[width=2.58333in]{images/assoc1.png}

and get \texttt{T∘T}; which can be further reduced to \texttt{T} by
applying \texttt{μ}. \texttt{IT} is the identity natural transformation
from \texttt{T} to \texttt{T}. You will often see the notation for this
type of horizontal composition \texttt{IT\ ∘\ μ} shortened to
\texttt{T∘μ}. This notation is unambiguous because it makes no sense to
compose a functor with a natural transformation, therefore \texttt{T}
must mean \texttt{IT} in this context.

We can also draw the diagram in the (endo-) functor category
\texttt{{[}C,\ C{]}}:

\includegraphics[width=1.73958in]{images/assoc2.png}

Alternatively, we can treat \texttt{T3} as the composition of
\texttt{T2∘T} and apply \texttt{μ∘T} to it. The result is also
\texttt{T∘T} which, again, can be reduced to \texttt{T} using μ. We
require that the two paths produce the same result.

\includegraphics[width=2.16667in]{images/assoc.png}

Similarly, we can apply the horizontal composition \texttt{η∘T} to the
composition of the identity functor \texttt{I} after \texttt{T} to
obtain \texttt{T2}, which can then be reduced using \texttt{μ}. The
result should be the same as if we applied the identity natural
transformation directly to \texttt{T}. And, by analogy, the same should
be true for \texttt{T∘η}.

\includegraphics[width=3.12500in]{images/unitlawcomp-1.png}

You can convince yourself that these laws guarantee that the composition
of Kleisli arrows indeed satisfies the laws of a category.

The similarities between a monad and a monoid are striking. We have
multiplication μ, unit η, associativity, and unit laws. But our
definition of a monoid is too narrow to describe a monad as a monoid. So
let's generalize the notion of a monoid.

\subsection{Monoidal Categories}\label{monoidal-categories}

Let's go back to the conventional definition of a monoid. It's a set
with a binary operation and a special element called unit. In Haskell,
this can be expressed as a typeclass:

\begin{verbatim}
class Monoid m where mappend :: m -> m -> m mempty :: m
\end{verbatim}

The binary operation \texttt{mappend} must be associative and unital
(i.e., multiplication by the unit \texttt{mempty} is a no-op).

Notice that, in Haskell, the definition of \texttt{mappend} is curried.
It can be interpreted as mapping every element of \texttt{m} to a
function:

\begin{verbatim}
mappend :: m -> (m -> m)
\end{verbatim}

It's this interpretation that gives rise to the definition of a monoid
as a single-object category where endomorphisms
\texttt{(m\ -\textgreater{}\ m)} represent the elements of the monoid.
But because currying is built into Haskell, we could as well have
started with a different definition of multiplication:

\begin{verbatim}
mu :: (m, m) -> m
\end{verbatim}

Here, the cartesian product \texttt{(m,\ m)} becomes the source of pairs
to be multiplied.

This definition suggests a different path to generalization: replacing
the cartesian product with categorical product. We could start with a
category where products are globally defined, pick an object \texttt{m}
there, and define multiplication as a morphism:

\begin{verbatim}
μ :: m × m -> m
\end{verbatim}

We have one problem though: In an arbitrary category we can't peek
inside an object, so how do we pick the unit element? There is a trick
to it. Remember how element selection is equivalent to a function from
the singleton set? In Haskell, we could replace the definition of
\texttt{mempty} with a function:

\begin{verbatim}
eta :: () -> m
\end{verbatim}

The singleton is the terminal object in \textbf{Set}, so it's natural to
generalize this definition to any category that has a terminal object
\texttt{t}:

\begin{verbatim}
η :: t -> m
\end{verbatim}

This lets us pick the unit ``element'' without having to talk about
elements.

Unlike in our previous definition of a monoid as a single-object
category, monoidal laws here are not automatically satisfied --- we have
to impose them. But in order to formulate them we have to establish the
monoidal structure of the underlying categorical product itself. Let's
recall how monoidal structure works in Haskell first.

We start with associativity. In Haskell, the corresponding equational
law is:

\begin{verbatim}
mu x (mu y z) = mu (mu x y) z
\end{verbatim}

Before we can generalize it to other categories, we have to rewrite it
as an equality of functions (morphisms). We have to abstract it away
from its action on individual variables --- in other words, we have to
use point-free notation. Knowning that the cartesian product is a
bifunctor, we can write the left hand side as:

\begin{verbatim}
(mu . bimap id mu)(x, (y, z))
\end{verbatim}

and the right hand side as:

\begin{verbatim}
(mu . bimap mu id)((x, y), z)
\end{verbatim}

This is almost what we want. Unfortunately, the cartesian product is not
strictly associative --- \texttt{(x,\ (y,\ z))} is not the same as
\texttt{((x,\ y),\ z)} --- so we can't just write point-free:

\begin{verbatim}
mu . bimap id mu = mu . bimap mu id
\end{verbatim}

On the other hand, the two nestings of pairs are isomorphic. There is an
invertible function called the associator that converts between them:

\begin{verbatim}
alpha :: ((a, b), c) -> (a, (b, c)) alpha ((x, y), z) = (x, (y, z))
\end{verbatim}

With the help of the associator, we can write the point-free
associativity law for \texttt{mu}:

\begin{verbatim}
mu . bimap id mu . alpha = mu . bimap mu id
\end{verbatim}

We can apply a similar trick to unit laws which, in the new notation,
take the form:

\begin{verbatim}
mu (eta (), x) = x mu (x, eta ()) = x
\end{verbatim}

They can be rewritten as:

\begin{verbatim}
(mu . bimap eta id) ((), x) = lambda ((), x) (mu . bimap id eta) (x, ()) = rho (x, ())
\end{verbatim}

The isomorphisms \texttt{lambda} and \texttt{rho} are called the left
and right unitor, respectively. They witness the fact that the unit
\texttt{()} is the identity of the cartesian product up to isomorphism:

\begin{verbatim}
lambda :: ((), a) -> a lambda ((), x) = x
\end{verbatim}

\begin{verbatim}
rho :: (a, ()) -> a rho (x, ()) = x
\end{verbatim}

The point-free versions of the unit laws are therefore:

\begin{verbatim}
mu . bimap id eta = lambda mu . bimap eta id = rho
\end{verbatim}

We have formulated point-free monoidal laws for \texttt{mu} and
\texttt{eta} using the fact that the underlying cartesian product itself
acts like a monoidal multiplication in the category of types. Keep in
mind though that the associativity and unit laws for the cartesian
product are valid only up to isomorphism.

It turns out that these laws can be generalized to any category with
products and a terminal object. Categorical products are indeed
associative up to isomorphism and the terminal object is the unit, also
up to isomorphism. The associator and the two unitors are natural
isomorphisms. The laws can be represented by commuting diagrams.

\includegraphics[width=3.12500in]{images/assocmon.png}

Notice that, because the product is a bifunctor, it can lift a pair of
morphisms --- in Haskell this was done using \texttt{bimap}.

We could stop here and say that we can define a monoid on top of any
category with categorical products and a terminal object. As long as we
can pick an object \texttt{m} and two morphisms μ and η that satisfy
monoidal laws, we have a monoid. But we can do better than that. We
don't need a full-blown categorical product to formulate the laws for μ
and η. Recall that a product is defined through a universal construction
that uses projections. We haven't used any projections in our
formulation of monoidal laws.

A bifunctor that behaves like a product without being a product is
called a tensor product, often denoted by the infix operator ⊗. A
definition of a tensor product in general is a bit tricky, but we won't
worry about it. We'll just list its properties --- the most important
being associativity up to isomorphism.

Similarly, we don't need the object \texttt{t} to be terminal. We never
used its terminal property --- namely, the existence of a unique
morphism from any object to it. What we require is that it works well in
concert with the tensor product. Which means that we want it to be the
unit of the tensor product, again, up to isomorphism. Let's put it all
together:

A monoidal category is a category \emph{C} equipped with a bifunctor
called the tensor product:

\begin{verbatim}
⊗ :: C × C -> C
\end{verbatim}

and a distinct object \texttt{i} called the unit object, together with
three natural isomorphisms called, respectively, the associator and the
left and right unitors:

\begin{verbatim}
αa b c :: (a ⊗ b) ⊗ c -> a ⊗ (b ⊗ c) λa :: i ⊗ a -> a ρa :: a ⊗ i -> a
\end{verbatim}

(There is also a coherence condition for simplifying a quadruple tensor
product.)

What's important is that a tensor product describes many familiar
bifunctors. In particular, it works for a product, a coproduct and, as
we'll see shortly, for the composition of endofunctors (and also for
some more esoteric products like Day convolution). Monoidal categories
will play an essential role in the formulation of enriched categories.

\subsection{Monoid in a Monoidal
Category}\label{monoid-in-a-monoidal-category}

We are now ready to define a monoid in a more general setting of a
monoidal category. We start by picking an object \texttt{m}. Using the
tensor product we can form powers of \texttt{m}. The square of
\texttt{m} is \texttt{m\ ⊗\ m}. There are two ways of forming the cube
of \texttt{m}, but they are isomorphic through the associator. Similarly
for higher powers of \texttt{m} (that's where we need the coherence
conditions). To form a monoid we need to pick two morphisms:

\begin{verbatim}
μ :: m ⊗ m -> m η :: i -> m
\end{verbatim}

where \texttt{i} is the unit object for our tensor product.

\includegraphics[width=3.12500in]{images/monoid-1.jpg}

These morphisms have to satisfy associativity and unit laws, which can
be expressed in terms of the following commuting diagrams:

\includegraphics[width=3.12500in]{images/assoctensor.jpg}

\includegraphics[width=3.12500in]{images/unitmon.jpg}

Notice that it's essential that the tensor product be a bifunctor
because we need to lift pairs of morphisms to form products such as
\texttt{μ\ ⊗\ id} or \texttt{η\ ⊗\ id}. These diagrams are just a
straightforward generalization of our previous results for categorical
products.

\subsection{Monads as Monoids}\label{monads-as-monoids}

Monoidal structures pop up in unexpected places. One such place is the
functor category. If you squint a little, you might be able to see
functor composition as a form of multiplication. The problem is that not
any two functors can be composed --- the target category of one has to
be the source category of the other. That's just the usual rule of
composition of morphisms --- and, as we know, functors are indeed
morphisms in the category \textbf{Cat}. But just like endomorphisms
(morphisms that loop back to the same object) are always composable, so
are endofunctors. For any given category \emph{C}, endofunctors from
\emph{C} to \emph{C} form the functor category \texttt{{[}C,\ C{]}}. Its
objects are endofunctors, and morphisms are natural transformations
between them. We can take any two objects from this category, say
endofunctors \texttt{F} and \texttt{G}, and produce a third object
\texttt{F\ ∘\ G} --- an endofunctor that's their composition.

Is endofunctor composition a good candidate for a tensor product? First,
we have to establish that it's a bifunctor. Can it be used to lift a
pair of morphisms --- here, natural transformations? The signature of
the analog of \texttt{bimap} for the tensor product would look something
like this:

\begin{verbatim}
bimap :: (a -> b) -> (c -> d) -> (a ⊗ c -> b ⊗ d)
\end{verbatim}

If you replace objects by endofunctors, arrows by natural
transformations, and tensor products by composition, you get:

\begin{verbatim}
(F -> F&apos;) -> (G -> G&apos;) -> (F ∘ G -> F&apos; ∘ G&apos;)
\end{verbatim}

which you may recognize as the special case of horizontal composition.

\includegraphics[width=2.65625in]{images/horizcomp.png}

We also have at our disposal the identity endofunctor \texttt{I}, which
can serve as the identity for endofunctor composition --- our new tensor
product. Moreover, functor composition is associative. In fact
associativity and unit laws are strict --- there's no need for the
associator or the two unitors. So endofunctors form a strict monoidal
category with functor composition as tensor product.

What's a monoid in this category? It's an object --- that is an
endofunctor \texttt{T}; and two morphisms --- that is natural
transformations:

\begin{verbatim}
μ :: T ∘ T -> T η :: I -> T
\end{verbatim}

Not only that, here are the monoid laws:

\includegraphics[width=1.90625in]{images/assoc.png}

\includegraphics[width=2.86458in]{images/unitlawcomp.png}

They are exactly the monad laws we've seen before. Now you understand
the famous quote from Saunders Mac Lane:

All told, monad is just a monoid in the category of endofunctors.

You might have seen it emblazoned on some t-shirts at functional
programming conferences.

\subsection{Monads from Adjunctions}\label{monads-from-adjunctions}

An
\href{https://bartoszmilewski.com/2016/04/18/adjunctions/}{adjunction},
\texttt{L\ ⊣\ R}, is a pair of functors going back and forth between two
categories \emph{C} and \emph{D}. There are two ways of composing them
giving rise to two endofunctors, \texttt{R\ ∘\ L} and \texttt{L\ ∘\ R}.
As per an adjunction, these endofunctors are related to identity
functors through two natural transformations called unit and counit:

\begin{verbatim}
η :: ID -> R ∘ L ε :: L ∘ R -> IC
\end{verbatim}

Immediately we see that the unit of an adjunction looks just like the
unit of a monad. It turns out that the endofunctor \texttt{R\ ∘\ L} is
indeed a monad. All we need is to define the appropriate μ to go with
the η. That's a natural transformation between the square of our
endofunctor and the endofunctor itself or, in terms of the adjoint
functors:

\begin{verbatim}
R ∘ L ∘ R ∘ L -> R ∘ L
\end{verbatim}

And, indeed, we can use the counit to collapse the \texttt{L\ ∘\ R} in
the middle. The exact formula for μ is given by the horizontal
composition:

\begin{verbatim}
μ = R ∘ ε ∘ L
\end{verbatim}

Monadic laws follow from the identities satisfied by the unit and counit
of the adjunction and the interchange law.

We don't see a lot of monads derived from adjunctions in Haskell,
because an adjunction usually involves two categories. However, the
definitions of an exponential, or a function object, is an exception.
Here are the two endofunctors that form this adjunction:

\begin{verbatim}
L z = z × s R b = s ⇒ b
\end{verbatim}

You may recognize their composition as the familiar state monad:

\begin{verbatim}
R (L z) = s ⇒ (z × s)
\end{verbatim}

We've seen this monad before in Haskell:

\begin{verbatim}
newtype State s a = State (s -> (a, s))
\end{verbatim}

Let's also translate the adjunction to Haskell. The left functor is the
product functor:

\begin{verbatim}
newtype Prod s a = Prod (a, s)
\end{verbatim}

and the right functor is the reader functor:

\begin{verbatim}
newtype Reader s a = Reader (s -> a)
\end{verbatim}

They form the adjunction:

\begin{verbatim}
instance Adjunction (Prod s) (Reader s) where counit (Prod (Reader f, s)) = f s unit a = Reader (\s -> Prod (a, s))
\end{verbatim}

You can easily convince yourself that the composition of the reader
functor after the product functor is indeed equivalent to the state
functor:

\begin{verbatim}
newtype State s a = State (s -> (a, s))
\end{verbatim}

As expected, the \texttt{unit} of the adjunction is equivalent to the
\texttt{return} function of the state monad. The \texttt{counit} acts by
evaluating a function acting on its argument. This is recognizable as
the uncurried version of the function \texttt{runState}:

\begin{verbatim}
runState :: State s a -> s -> (a, s) runState (State f) s = f s
\end{verbatim}

(uncurried, because in \texttt{counit} it acts on a pair).

We can now define \texttt{join} for the state monad as a component of
the natural transformation μ. For that we need a horizontal composition
of three natural transformations:

\begin{verbatim}
μ = R ∘ ε ∘ L
\end{verbatim}

In other words, we need to sneak the counit ε across one level of the
reader functor. We can't just call \texttt{fmap} directly, because the
compiler would pick the one for the \texttt{State} functor, rather than
the \texttt{Reader} functor. But recall that \texttt{fmap} for the
reader functor is just left function composition. So we'll use function
composition directly.

We have to first peel off the data constructor \texttt{State} to expose
the function inside the \texttt{State} functor. This is done using
\texttt{runState}:

\begin{verbatim}
ssa :: State s (State s a) runState ssa :: s -> (State s a, s)
\end{verbatim}

Then we left-compose it with the counit, which is defined by
\texttt{uncurry\ runState}. Finally, we clothe it back in the
\texttt{State} data constructor:

\begin{verbatim}
join :: State s (State s a) -> State s a join ssa = State (uncurry runState . runState ssa)
\end{verbatim}

This is indeed the implementation of \texttt{join} for the
\texttt{State} monad.

It turns out that not only every adjunction gives rise to a monad, but
the converse is also true: every monad can be factorized into a
composition of two adjoint functors. Such factorization is not unique
though.

We'll talk about the other endofunctor \texttt{L\ ∘\ R} in the next
section.

Next: \href{https://bartoszmilewski.com/2017/01/02/comonads/}{Comonads}.
